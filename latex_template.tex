%
% Presenation document
%
\documentclass{beamer}
\usepackage{graphicx}
\graphicspath{ {../../pictures/} }
\usepackage[english]{babel}
\usepackage{amsmath}
\usepackage[
  backend=biber,
  style=nature,
  citestyle=authoryear,
  maxcitenames=3,
  maxbibnames=15
]{biblatex}
\usepackage{array}
\newcolumntype{L}{>{\centering\arraybackslash}m{3cm}}
\addbibresource{../../papers/bib.bib}
\usefonttheme{structuresmallcapsserif}
%\usepackage{pdfpages}
\usetheme{Singapore}
\usecolortheme{dove}
%
\setbeamertemplate{footline}{%
\raisebox{5pt}{\makebox[\paperwidth]{\hfill\makebox[20pt]{\color{gray}
\scriptsize\insertframenumber}}}\hspace*{5pt}}
\beamertemplatenavigationsymbolsempty
\usepackage[overridenote]{pdfpc} % For including the notes in the presentation using pdfpc (not in extra slides).
\begin{document}
%
%
\title{My super influential study}

\author{My name}

\institute{My team, group, institute \\
Supervisor: My supervisor}

\date{November 08, 2020}


\begin{frame}[plain]
  \titlepage
  \makebox[0.9\paperwidth]{%
    \includegraphics[height=1cm,keepaspectratio]{artix.png}%
    \hfill%
    \includegraphics[height=1cm,keepaspectratio]{arch-2.jpg}%
    }%

\end{frame}



%
\begin{frame}
  \frametitle{Table of Contents}
  \tableofcontents
\end{frame}
%

\section{Introduction}

  \begin{frame}
    \frametitle{Table of Contents}
    \tableofcontents[currentsection]
  \end{frame}

  \begin{frame}
    \frametitle{The objective}
    \centering
    The objectives of the current study are... \pause
    \newline
    \newline

    \(\overset{\curvearrowleft}{\textbf{DNA}}
    \rightleftarrows \overset{\curvearrowleft}{\textbf{RNA}}
    \rightarrow \textbf{Protein}\)

    \pause
    \begin{columns}

      \column{.5\textwidth}
      \begin{itemize}
        \item Replication \pause
        \item Transcription \pause
        \item Translation \pause
      \end{itemize}

      \column{.5\textwidth}
      \begin{itemize}
        \item RNA splicing \pause
        \item RNA export
      \end{itemize}
    \end{columns}

  \end{frame}




  \begin{frame}
    \centering
    \textbf{Important observation}
  \end{frame}


  \begin{frame}
    \frametitle{Methods}

    \centering
    My methods \\ [0.5ex]
    \textbf{Normal} distribution
    \includegraphics[width=\textwidth]{math_map.jpg}
    \footcite{Gaussian_derivation}

  \end{frame}

  \begin{frame}
    \frametitle{The dataset}

    \centering
    \includegraphics[width=0.9\textwidth]{artix.png}
    \vfill

  \end{frame}

  \begin{frame}
    \frametitle{questions}
    \begin{itemize}
      \item <1-> Question 1
      \item <2-> Question 2
      \item <3> Question 3
      \item <4-> Question 4
    \end{itemize}
  \end{frame}

  \begin{frame}
    \frametitle{questions}
    \only<1>{Question 5}
    \only<2>{Question 6}
    \only<3>{Question 7}
    \only<4>{Question 8}
  \end{frame}



\section{Progress}
\subsection{Problem formalization}

  \begin{frame}
    \frametitle{Table of Contents}
    \tableofcontents[currentsection]
  \end{frame}

  \begin{frame}
    \frametitle{Mathematical formulation}
    \begin{equation} \label{eq1}
    \begin{split}
    A & = \frac{\pi r^2}{2} \\
     & = \frac{1}{2} \pi r^2
    \end{split}
    \end{equation}

  \end{frame}


  \begin{frame}
    \frametitle{mathematical formulation}
    \begin{align*}
      x&=y           &  w &=z              &  a&=b+c\\
      2x&=-y         &  3w&=\frac{1}{2}z   &  a&=b\\
      -4 + 5x&=2+y   &  w+2&=-1+w          &  ab&=cb
    \end{align*}
  \end{frame}


  \begin{frame}
    \begin{figure}[h]
      \centering
      \includegraphics[width=0.8\textwidth]{DNA.png}
      \footcite{WATSON_1953}
    \end{figure}

  \end{frame}




\subsection{More data}

  \begin{frame}
    \frametitle{More data}
    \pause
    \begin{figure}[h]
      \centering
      \includegraphics[height=0.8\textheight]{DNA_AA.png}
      \footcite{Li_2011}
    \end{figure}
  \end{frame}

  \begin{frame}
    \frametitle{Results}
    My super significant outcome
    \begin{columns}

      \column{.5\textwidth}
      \pause
      \begin{figure}[h]
        \centering
        Title 1
        \includegraphics[width=0.9\linewidth]{arch-1.jpg}
        \caption{1 Arch1}
        \label{fig:1}
      \end{figure} \pause


      \column{.5\textwidth}
      \begin{figure}[h]
        \centering
        Title 2
        \includegraphics[width=0.9\linewidth]{arch-3.jpg}
        \caption{2 Arch2}
        \label{fig:2}
      \end{figure}

    \end{columns}

  \end{frame}

  \begin{frame}
    You see the same result in Figure \ref{fig:1} as in Figure \ref{fig:2}
  \end{frame}


  \begin{frame}
    \frametitle{Results}
    \begin{table}
      \centering
      \begin{tabular}{||c|L|L||}
        \hline
        & Average \(p_{i}\) &
        Median \(q_{i}\)\\ [0.5ex]
        \hline\hline
        Mode & 0.713 & 0.287 \\
        Estimates & 0.677 & 0.323 \\
        \hline
      \end{tabular}
      \label{tab:2}
    \end{table}
  \end{frame}


\section{Future Plans}

  \begin{frame}
    \frametitle{Table of Contents}
    \tableofcontents[currentsection]
  \end{frame}


  \begin{frame}
    \frametitle{future work}

    \begin{enumerate}
      \item<3-> Next step 1
      \item<2-> Next step 2
      \item<1-> Next step 3
    \end{enumerate}

    \note{My notes...}
  \end{frame}


\section{Bibliography}
\begin{frame}
  \frametitle{Table of Contents}
  \tableofcontents[currentsection]
\end{frame}

\begin{frame}[allowframebreaks]
  \frametitle{References}

  \setbeamertemplate{bibliography item}{\insertbiblabel}
  \printbibliography

\end{frame}


\appendix


\section{Acknowledgements}

\begin{frame}
  \frametitle{Acknowledgements}
  \centering
  \underline{Special Thanks} \\ [0.5ex]
  Collaborators \\ [10ex]
  \pause
  \textit{Thank you for your attention!}
\end{frame}



\section{Supplementary material}

\begin{frame}[allowframebreaks]
  \centering
  \textbf{Even} \textit{more} \underline{stuff}
\end{frame}


\end{document}
